\documentclass[10pt,a4paper]{article}
\usepackage[utf8]{inputenc}
\usepackage[english]{babel}
\usepackage[T1]{fontenc}
\usepackage{amsmath}
\usepackage{amsfonts}
\usepackage{amssymb}
\usepackage{lmodern}
\usepackage{fancyvrb}
\usepackage{tikz}

\newcommand{\version}{\input{../../version.txt}}

\newcommand{\command}[1]{%
\noindent \fbox{%
   \begin{minipage}{\textwidth}
      #1
   \end{minipage}%
}
}

\newcommand{\sample}[1]{%
\vspace{0.2cm}Exemple :\\
\command{#1}
}

\author{Sebastien CAUX}
\title{gpnode doc \version}

\begin{document}
\maketitle
\section{Introduction}
gpnode is a command line tool that permits to create and manage a node project in GPStudio toolchain.

A node in GPStudio is a physical node, it can be a smart camera or a sensor.

\section{Use}
gpnode always takes the project in the current directory, so you only can have one project per directory. A node project file have the '.node' extension.

At the very beginning, you need to create a project with the \emph{newproject} command. After that, you can use all the commands set on this project.

Please read the tutorial 'GPStudio command line quick start' to learn how to use this tool.

\section{Commands}
\subsection{project}
\subsubsection{newproject}
\command{\textbf{gpnode} \textbf{newproject} -n <project-name>}

Create a project file inside the current directory named '\emph{<project-name>}.node'.

\sample{\textbf{gpnode} \textbf{newproject} -n project1}

\subsubsection{setboard}
\command{\textbf{gpnode} \textbf{setboard} -n <board-name>}

Specify the name of the used board for the node. You need to specify a board before setting up any io in the project.

\sample{\textbf{gpnode} \textbf{setboard} -n dreamcam\_c3}

\subsubsection{generate}
\command{\textbf{gpnode} \textbf{generate} [-o <dir>]}

Generate all files needed for the specified toolchain and Makefile. After that, you just need to call 'make compile' to compile the project with specific tools needed by the node.

\sample{\textbf{gpnode} \textbf{generate} -o build/}

\subsection{io}
\subsubsection{addio}
\command{\textbf{gpnode} \textbf{addio} -n <io-name>}

Add IP support in the project to manage \emph{<io-name>}. \emph{<io-name>} must be define in the board file definition.

\sample{\textbf{gpnode} \textbf{addio} -n mt9}

\subsubsection{delio}
\command{\textbf{gpnode} \textbf{delio} -n <io-name>}

Remove io support named \emph{<io-name>}.

\sample{\textbf{gpnode} \textbf{delio} -n mt9}

\subsubsection{showio}
\command{\textbf{gpnode} \textbf{showio}}

Print the list of io support in the current project.

\sample{\textbf{gpnode} \textbf{showio}\\ios :\\ + mt9\\ + usb}

\subsection{process}
\subsubsection{addprocess}
\command{\textbf{gpnode} \textbf{addprocess} -n \emph{<process-name>} -d \emph{<driver-name>}}

Add a process named \emph{<process-name>} as an instance of \emph{<driver-name>} IP in the library or the project dir.

\sample{\textbf{gpnode} \textbf{addprocess} -n process1 -d gradienthw}

\subsubsection{delprocess}
\command{\textbf{gpnode} \textbf{delprocess} -n <process-name>}

Remove process \emph{<process-name>} and all the connections to or from it.

\subsubsection{showprocess}
\command{\textbf{gpnode} \textbf{showprocess}}

Print the list of process in the current project.

\sample{\textbf{gpnode} \textbf{showprocess}\\ios :\\ + process1\\ + process2}

\subsubsection{showblock}
\command{\textbf{gpnode} \textbf{showblock}}

Print the list of process and io in the current project.

\sample{\textbf{gpnode} \textbf{showblock}\\blocks :\\ + mt9\\ + usb\\ + process1\\ + process2}

\subsection{block attributes}
\subsubsection{renameblock}
\command{\textbf{gpnode} \textbf{renameblock} -n <block-name> -v <new-name>}

Rename a process block.

\sample{\textbf{gpnode} \textbf{renameblock} -n process1 -v first\_gradient}

\subsubsection{setproperty}
\command{\textbf{gpnode} \textbf{setproperty} -n <property-name> -v <default-value>}

Define a default value \emph{<default-value>} to the property \emph{<property-name>}.

\subsubsection{setparam}
\command{\textbf{gpnode} \textbf{setparam} -n <param-name> -v <value>}

Set the value \emph{<value>} to the param \emph{<param-name>}.

\subsubsection{setclock}
\command{\textbf{gpnode} \textbf{setclock} -n <clock-name> -v <frequency>}

Define the clock frequency \emph{<frequency>} to the clock \emph{<clock-name>}.

\subsubsection{setflowsize}
\command{\textbf{gpnode} \textbf{setflowsize} -n <flow-name> -v <flow-size>}

define the flow size \emph{<flow-size>} to the flow \emph{<flow-name>}

\subsection{flow interconnect}
\subsubsection{connect}
\command{\textbf{gpnode} \textbf{connect} -f <flow-out> -t <flow-in> [-s <bit-shift>]}

Add a flow connection between a flow out \emph{<flow-out>} (ex: mt9.out) to a flow in \emph{<flow-in>} with a bit shift \emph{<bit-shift>} ("lsb" or "msb").

\subsubsection{unconnect}
\command{\textbf{gpnode} \textbf{unconnect} -f <flow-out> -t <flow-in>}

Remove a flow connection between a flow out \emph{<flow-out>} (ex: mt9.out) to a flow in \emph{<flow-in>}.

\subsubsection{showconnects}
\command{\textbf{gpnode} \textbf{showconnects}}

Print the list of flow connections in the current project.

\subsection{clock interconnect}
\subsubsection{setclockdomain}
\command{\textbf{gpnode} \textbf{setclockdomain} -n <domain-name> -v <frequency>}

Define a clock frequency \emph{<frequency>} the the clock domain \emph{<domain-name>}.

\subsubsection{showclockdomain}
\command{\textbf{gpnode} \textbf{showclockdomain}}

Print the list of clock domains in the current project.

%\subsection{lists}
%\subsubsection{listavailableio}
%\begin{Verbatim}[frame=single]
%gpnode listavailableio
%\end{Verbatim}
%
%\subsubsection{listio}
%\begin{Verbatim}[frame=single]
%gpnode listio
%\end{Verbatim}
%
%\subsubsection{listavailableprocess}
%\begin{Verbatim}[frame=single]
%gpnode listavailableprocess
%\end{Verbatim}
%
%\subsubsection{listprocess}
%\begin{Verbatim}[frame=single]
%gpnode listprocess
%\end{Verbatim}
%
%\subsubsection{listword}
%\begin{Verbatim}[frame=single]
%gpnode listword
%\end{Verbatim}

\end{document}
