\documentclass[10pt,a4paper]{article}
\usepackage[utf8]{inputenc}
\usepackage[english]{babel}
\usepackage[T1]{fontenc}
\usepackage{amsmath}
\usepackage{amsfonts}
\usepackage{amssymb}
\usepackage{lmodern}
\usepackage{fancyvrb}
\usepackage{tikz}
\usepackage{verbatim}
\usepackage{indentfirst}

\newcommand{\version}{\IfFileExists{../../version.txt}
{\input{../../version.txt}}
{\input{../../../version.txt}}
}

\newcommand{\command}[1]{%
\noindent \fcolorbox{black}{white}{%
   \begin{minipage}{\textwidth}%
      #1
   \end{minipage}%
}
}

\newcommand{\sample}[1]{%
\noindent \fcolorbox{black}{lightgray}{%
   \begin{minipage}{\textwidth}%
      #1
   \end{minipage}%
}
}

\newcommand{\sampletitle}[1]{%
\vspace{0.2cm}Exemple :\\%
\noindent \sample{#1}%
}

\newcommand{\samplecomment}[1]{%

\textit{#1}
}

\author{Sebastien CAUX}
\title{gpblock reference documentation \version}

\begin{document}
\maketitle
\section{Introduction}
gpblock is a command line tool that permits to create new process block.

\section{Use}
gpblock always takes the project in the current directory, so you only can have one project per directory. A node project file have the `.node' extension.

At the beginning, you need to create a project with the \emph{newproject} command. After that, you can use all the commands set on this project.

Please read the tutorial `GPStudio command line quick start' to learn how to use this tool.

Under linux, you have a completion script to help you writing commands.

\section{Commands}
\subsection{project}
\subsubsection{newproject}
\command{\textbf{gpnode} \textbf{newproject} -n <project-name>}

Create a project file inside the current directory named `\texttt{<project-name>}.node'.

\paramcommand{
-n & project name without space & project1 \\ 
}

\begin{sampletitle}
> \textbf{gpnode} \textbf{newproject} -n project1
\end{sampletitle}
\samplecomment{Create a new project named \texttt{project1}. After that, you have a file project \texttt{project1.node} in the current directory.}

\seealso{\textbf{gpnode setclockdomain}}

\end{document}
