%------------------------ Packages ------------------------
\documentclass[12pt,a4paper]{article}
\usepackage[latin1]{inputenc}
\usepackage[T1]{fontenc}
\usepackage[pdftex]{graphicx}
\usepackage{float}
\usepackage{amsmath}
\usepackage{amssymb}
\usepackage[FIGTOPCAP]{subfigure}
\usepackage{color}
\usepackage[hidelinks]{hyperref}

\newcommand{\version}{\IfFileExists{../../version.txt}
{\input{../../version.txt}}
{\input{../../../version.txt}}
}

\newcommand{\command}[1]{%
\noindent \fcolorbox{black}{white}{%
   \begin{minipage}{\textwidth}%
      #1
   \end{minipage}%
}
}

\newcommand{\sample}[1]{%
\noindent \fcolorbox{black}{lightgray}{%
   \begin{minipage}{\textwidth}%
      #1
   \end{minipage}%
}
}

\newcommand{\sampletitle}[1]{%
\vspace{0.2cm}Exemple :\\%
\noindent \sample{#1}%
}

\newcommand{\samplecomment}[1]{%

\textit{#1}
}

\begin{document}

\begin{center}
\textbf{\huge  \underline{Scharr operator}}
\end{center}
\vspace{0.5cm}


The Scharr operator is used for edge detection algorithms. It is a discrete differentiation operator who computes, at each point in the image, an approximation of the horizontal and vertical gradient of the image intensity function.\\


\begin{figure}[h!]
\centering
\begin{tikzpicture}
\node[block,rectangle,minimum height=2cm,minimum width=3cm] (bloc) {\textbf{Scharr}};

\path[connect,<-] ([yshift=0.0cm]bloc.west) -- node{/} node[above]{in} ++(-2cm,0);

\path[connect,->] ([yshift=0.0cm]bloc.east) -- node{/} node[above]{out} ++(2cm,0);
 ([xshift=0.5cm,yshift=-0.6cm]bloc.north);

\end{tikzpicture}
\end{figure}

\vspace{0.5cm}

\begin{figure}[!h]
\centering
\subfigure[Initial image]{
\includegraphics[width=5cm]{scharr1.png}}
\hspace{2cm}
\subfigure[Image with Scharr operator]{
\includegraphics[width=5cm]{scharr2.png}}
\end{figure}

\section*{Properties}
\properties{
enable & bool & Enable the processing \\ 
}

\vspace{0.5cm}

\section*{Constants}

\constants{
LINE\_WIDTH\_MAX & Maximum line size of the image : 1280 pixels  \\ 
CLK\_PROC\_FREQ & Frequency clock of the process \\
IN\_SIZE & Size of the input flow : 1 byte \\
OUT\_SIZE & Size of the output flow : 1 byte  \\
WEIGHT\_SIZE & Size of the Kernel : 1 byte  \\
}

\newpage

\section*{Equivalence}
\subsection*{Matlab}

\lstset{language=Matlab}
\begin{lstlisting}
No available with Matlab
\end{lstlisting}

\url{https://nl.mathworks.com/help/images/ref/edge.html}


\subsection*{OpenCV}

\lstset{language=C++}
\begin{lstlisting}
void cv::Scharr 	
	(
	InputArray  	src,
	OutputArray  	dst,
	int  		ddepth,
	int  		dx,
	int  		dy,
	double  	scale = 1,
	double  	delta = 0,
	int  		borderType = BORDER_DEFAULT 
	) 		

\end{lstlisting}

\url{http://docs.opencv.org/}

\section*{Mathematical formalism}

The operator uses two $3X3$ kernels which are convolved with the original image to compute approximations of the horizontal and vertical derivatives. \\

If we define $I$ as the source image, $\bigtriangledown_x$ and $\bigtriangledown_y$ as the horizontal and vertical derivative approximation respectively. $\bigtriangledown_x$ and $\bigtriangledown_y$ are written as follows :\\

\begin{equation}\label{eq1}
\bigtriangledown_x = \begin{pmatrix}
-3 & 0 & 3 \\ 
-10 & 0 & 10 \\ 
-3 & 0 & 3
\end{pmatrix}*I
\end{equation}

\vspace{0.5cm}

\begin{equation}\label{eq2}
\bigtriangledown_y = \begin{pmatrix}
-3 & -10 & -3 \\ 
0 & 0 & 0 \\ 
3 & 10 & 3
\end{pmatrix}*I
\end{equation}

\vspace{0.5cm}
\newpage
We can develop (\ref{eq1}) and (\ref{eq2}) as follows :\\

\begin{equation}\label{eq3}
\begin{matrix}
\bigtriangledown_x = \begin{pmatrix}
-3 & 0 & 3 \\ 
-10 & 0 & 10 \\ 
-3 & 0 & 3
\end{pmatrix}*I  = \begin{pmatrix}
-3 & 0 & 3 \\ 
-10 & 0 & 10 \\ 
-3 & 0 & 3
\end{pmatrix}* \begin{pmatrix}
i_{00} & i_{01} & i_{02} \\ 
i_{10} & i_{11} & i_{12}\\ 
i_{20} & i_{21} & i_{22}
\end{pmatrix}  \\ 
& & \\
 =  -3i_{00}-10i_{10}-3i_{20}+3i_{02}+10i_{12}+3i_{22} \\ 
\end{matrix}
\end{equation}

\vspace{1cm}

\begin{equation}\label{eq4}
\begin{matrix}
\bigtriangledown_y = \begin{pmatrix}
-3 & -10 & -3 \\ 
0 & 0 & 0 \\ 
3 & 10 & 3
\end{pmatrix}*I  = \begin{pmatrix}
-3 & -10 & -3 \\ 
0 & 0 & 0 \\ 
3 & 10 & 3
\end{pmatrix}* \begin{pmatrix}
i_{00} & i_{01} & i_{02} \\ 
i_{10} & i_{11} & i_{12}\\ 
i_{20} & i_{21} & i_{22}
\end{pmatrix}  \\ 
& & \\
 =  -3i_{00}-10i_{01}-3i_{02}+3i_{20}+10i_{21}+3i_{22} \\ 
\end{matrix}
\end{equation}
 
\vspace{0.5cm}

In each point, we give an approximation of the gradient norm by the equation (\ref{eq5})

\begin{equation}\label{eq5}
\bigtriangledown = \sqrt{\bigtriangledown_x^2 + \bigtriangledown_y^2 } \simeq \bigtriangledown_x + \bigtriangledown_y
\end{equation}

\vspace{0.5cm}

Thus, we obtain :

\begin{equation}\label{eq6}
\bigtriangledown \simeq -6i_{00}-10i_{01}-10i_{10}+10i_{21}+10i_{12}+6i_{22}
\end{equation}
 
\vspace{0.25cm}
 
And finally in matrix form : 

\begin{equation}\label{eq7}
\kappa_S  \simeq \begin{pmatrix}
-6 & -10 & 0\\ 
-10 & 0 & 10\\ 
 0 & 10 & 6
\end{pmatrix}
\end{equation}

\end{document}